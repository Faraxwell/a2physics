\section*{Practical Issues}
%\markboth{Practical Issues}{Practical Issues}
%\addcontentsline{toc}{section}{Practical Issues}

\subsection*{Last Update: \today}

\subsection*{Contacts}
\textbf{Email}: \url{colin-young@live.com}

The latest update can be found via: \url{https://github.com/yuhao-yang-cy/a2physics}

%\item \textbf{Office Hours}
%
%There is no specific office hour. You can try to find me at my office 9am - 5pm, Monday to Friday. If you have questions about the course, I also recommend you to catch me right after any of my lectures.



\subsection*{About the Lecture Notes}

This is a set of very concise lecture notes written for CIE A-level Physics (syllabus code 9702). Presumably the target audience of the notes are students studying the relevant course.

I have been teaching A-Level courses for many years, and I have been planning to typeset my collection of handwritten notes with \LaTeX \phantom{ }not long after my teaching career started. The project never appeared to get any close to completion for many years, so I found joy when I eventually finished this project. 

These notes are supposed to be self-contained. I believe I have done my best to make the lecture notes reflect the spirit of the syllabus set by the Cambridge International Examination Board. Apart from the essential derivations and explanations, I also included a handful of worked examples and problem sets, so that you might get some rough idea about the styles of questions that you might encounter in the exams. If you are a student studying this course, I believe these notes could help you get well-prepared for the exams.

Throughout the notes, key concepts are marked red, key definitions and important formulas are boxed. But to be honest, the main reason that I wrote up these notes was not to serve any of my students, but just to give myself a goal. Since physics is such a rich and interesting subject, I cannot help sharing a small part of topics beyond the syllabus that I personally find interesting. If you see anything beyond the syllabus, they usually show up in the footnotes and are labelled with a star sign.

Note that there will be an update for the syllabus since 2022. I have plans to update these notes so that they are tailored for the new syllabus. I am glad to see some introductory astronomy and cosmology being added into the new syllabus, so hopefully you will see two new chapters coming in the near future.

Also very importantly, I am certain that there are tons of typos in the notes. If you spot any errors, please let me know.


\subsection*{Literature}

I borrow heavily from the following resources:

\begin{itemize}
\item[-] Cambridge International AS and A Level Physics Coursebook, by \textit{David Sang, Graham Jones, Richard Woodside} and \textit{Gurinder Chadha}, Cambridge University Press

\item[-] International A Level Physics Revision Guide, by \textit{Richard Woodside}, Hodder Education

\item[-] Longman Advanced Level Physics, by \textit{Kwok Wai Loo},	Pearson Education South Asia

\item[-] Past Papers of Cambridge Internation A-Level Physics Examinations

\item[-] HyperPhysics Website: \url{http://hyperphysics.phy-astr.gsu.edu/hbase/index.html}

\item[-] Wikipedia Website: \url{https://en.wikipedia.org}
\end{itemize}

\subsection*{Copyright}

This work is offered under a \textbf{CC BY-NC} (Creative Commons Attribution-Non-Commercial) license. You may remix, adapt, and build upon this work, as long as the attribution is given and the new work is non-commercial.