\documentclass[a4paper,12pt]{article}
\usepackage[svgnames]{xcolor}
\usepackage{tikz}
\usepackage{pgfplots}
%\usetikzlibrary{calc,fadings}
\usepackage[active,tightpage]{preview}
\PreviewEnvironment{tikzpicture}
\setlength\PreviewBorder{5pt}

\begin{document}
\thispagestyle{empty}

\tikzset{
  ring shading/.code args={from #1 at #2 to #3 at #4}{
    \pgfmathsetmacro{\inner}{25*#2/#4}
    \pgfmathsetmacro{\middle}{\inner/2+12.5}
    \pgfdeclareradialshading{ring}{\pgfpoint{0cm}{0cm}}%
    { color(0bp)=(white);
      color(\inner bp)=(#1);
      color(\middle bp)=(white);
      color(25bp)=(#3);
      color(50bp)=(black)}
    \pgfkeysalso{/tikz/shading=ring} } }
    
\begin{tikzpicture}[scale=1.5]
\draw[gray,fill] (-3.6,-3.6) rectangle (3.6,3.6);
\shade [inner color = white, outer color = gray, shading=radial] (0,0) circle (0.5);
\shade[even odd rule, ring shading={from gray at 1 to gray at 1.25}] (0,0) circle (1) (0,0) circle (1.25);
\shade[even odd rule, ring shading={from gray at 2 to gray at 2.16}] (0,0) circle (2) (0,0) circle (2.16);
\shade[even odd rule, ring shading={from gray at 2.4 to gray at 2.5}] (0,0) circle (2.4) (0,0) circle (2.5);
\shade[even odd rule, ring shading={from gray at 2.88 to gray at 3}] (0,0) circle (2.88) (0,0) circle (3);
\shade[even odd rule, ring shading={from gray at 3.2 to gray at 3.28}] (0,0) circle (3.2) (0,0) circle (3.28);
\end{tikzpicture}


\end{document}
